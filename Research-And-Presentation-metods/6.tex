\documentclass [10pt,a4paper]{book}

\begin{document}

\begin{flushleft}
\textbf{6  CHAPTER ONE}
\end{flushleft}
very moment. However, asynchronicity has long been confounded with text literacy.
Now we realize that text-based communication, supported either asynchronously or in
real time (as practiced in ICQ—an online instant messaging program, MOOs—Mud
Object Oriented, MUDs—Multi-User Dungeons, Palaces, and other Net-based chat
systems), is but one form of communication. In an advanced, Net-based context, voice,
sound, and video become as easily formatted, stored, and retrieved as text. Already,
early versions of asynchronous voice conferencing (for example, see www.wimba.com)
and asynchronous “virtual people speaking your email” animations of voice messaging
(i.e., hetp://www.lifefx.com) are becoming available in addition to synchronous audio
and video conferencing.
Because the Net so aptly supports both synchronous and asynchronous commu-
nication, it should be no surprise that ¢-research utilizes this capability to provide a
wide variety of research methods and tool capacities. Research applications can be cus-
tomized to take advantage of either synchronous or asynchronous formats—or both.
For example, online focus groups allow the researcher to gather groups of subjects
from widely disbursed geographic locations, These groups can be conducted synchro-
nously using voice or text formats so that instant feedback is provided to both
researchers and participants, and the immediate presence can be used to build common
understandings and ideas. Alternatively they can be conducted asynchronously, per-
mitting reflective interactions that are not dominated by the participants who think
and communicate most quickly.
E-research also utilizes the distributed data and information processing capacity
of the Net, Stand-alone data processing applications (including statistics programs,
registration systems, and programs that monitor network activity) are all becoming
“Net-enabled” and thereby can be applied to locations and times that are noncontin-
gent with the behavior or process being studied. Thus, e-researchers are able to use
research tools, monitor activity, and collect data without traveling long distances or
coordinating local time schedules.
E-research permits the exploration of new fields of knowledge. As more social
and economic interaction takes place on the networks, new fields of human endeavor
are created. Researchers can now study the ways in which students learn online or how
online education and civic groups make decisions and conduct business. These new
human activities grow in economic and political importance daily. These fields of
study are not readily accessible to researchers who cannot access or who lack the skills
to proficiently use the Net. Thus, this text is a guide that can be used for both instruc-
tion and motivation to acquire and effectively use the new tools and techniques of net-
worked research.
If, as Benedikt (1991) argues, cyberspace “has a geography, a physics, a nature
and a rule, of human law” (p. 123), then obviously it is an environment that can pro-
vide insight into human behavior and nature, through examination of the cultural and
sociological constructs that humans create within this context. Thus, cyberspace as an
evolving and extremely intricate human context attracts the researcher. It is unclear
how many of the research tools that have been developed, tested, and normed in real
communities will be as useful in virtual contexts. Likely, existing tools will need to be
modified to maximize their usefulness in this new milieu. Moreover, it is certain that

\end{document}
