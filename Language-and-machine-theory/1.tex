\documentclass [10pt,a4paper]{book}

\begin{document}

\begin{flushleft}

\textbf{Basic Terminology}
\end{flushleft}
\begin{flushleft}

\textbf{Introduction}
\end{flushleft}
In the automata theory, we have to deal with some mathematical preliminaries. As examples in finite automata and finite state machine the knowledge of set theory is necessary, in grammar and language scction we need the basic knowledge of alphabet, string, and substring, and in the regular expression chapter we need the concept of prefix, suffix, etc. The knowledge of the basic operations on a set such as union, intersection, difference, Cartesian product, power set, and concatenation product are required throughout the syllabus of formal language and automata theory.
For this reason, in this chapter, we shall discuss some basic terminologies related to mathematics which are required for automata theory.

\begin{flushleft}

\textbf{1.1 Basics of String}
\end{flushleft}


A string has some features as follows:


\begin{itemize}
\item

\textbf{Symbol:} A symbol is a user-defined entity.
\item

\textbf{Alphabet:} An alphabet is a finite set of symbols denoted by $\Sigma$ in automata. An alphabet is a set of symbols used to construct a language. As an example, \{0,1\} is a binary alphabet, and $\{\mathrm{A} \ldots \ldots, Z$ is an alphabet set for the English language.
\item

\textbf{String:} A string is defined as a scquence of symbols of finite length. A string is denoted by w in automata. As an example, 000111 is a binary string. (The length of a string w is denoted by $\mid \mathrm{w}$. For the previous case, $|\mathrm{w}|=|000111|=6$.)

\item
\textbf{Prefix:} A prefix of a string is the string formed by taking any number of symbols of the string.

\textbf{Example:} Let us take a string $\mathrm{w}=0111$. For the particular string. $\lambda, 0,01,011,$ and 0111 are prefixes of the string 0111 . For a string of length $\mathrm{n},$ there are $\mathrm{n}+1$ mumber of prefixes.

\item

\textbf{Proper prefix:} For a string, any prefix of the string other than the string itself is called as the proper prefix of the string.

\textbf{Example:} For the string $w=0111,$ the proper prefixes are $\lambda, 0,01,$ and 011

\item

\textbf{Suffix:} A suffix of a string is formed by taking any number of symbols from the end of the string.

\textbf{Example:} Let us take a string $w=0110$. For the particular string. $\lambda, 0,10,110,$ and 0110 are suffixes of the string 0110 . For a string of length $\mathrm{n},$ there are $\mathrm{n}+1$ number of suffixes.

\item

\textbf{Proper suftix:} For a string, any suflix of the string other than the string itself is called as the proper suffix of the string.

\textbf{Examplez} For the string $w=0110$, the proper suffixes are $\lambda, 0,10,$ and $110 .$


\end{itemize}

\end{document}
