\documentclass [10pt,a4paper]{book}

\begin{document}

\begin{flushleft}
\textbf{2 Introduction to Automata Theory, Formal Languages and Computation}
\end{flushleft}
\begin{itemize}
\item
\textbf{ Substring:} A substring of a string is defined as a string formed by taking any number of symbols of the string.

\textbf{Example:} For the string $w=012,$ the substrings are $\lambda, 0,1,2,01,12,$ and 012
\end{itemize}

\textbf{1.2 Basics of Set Theory}

A set is a well-defined collection of objects. The objects used for constructing

a set are called elements or the members of the set.


A set has some features as follows:
\begin{itemize}
\item
A set is a collection of objects. This collection is regarded as a single entity.
\item
A set is comprised of distinct elements. If an clement, say "a',is in set $S$, then it is denoted as a $\in$ S.
\item
A set has a well-defined boundary. If $\mathrm{S}$ is a set and 'a' is any element, then depending on the
properties of ‘a’,it can be said whether a € S ora¢ S.
\item
A sct is characterized by its property. In general, if $\mathrm{p}$ is the defined property for the elements of $\mathrm{S}$,then $\mathrm{S}$is denoted as $\mathrm{S}=\{\mathrm{a}:$ a has the property $\mathrm{p}\}$
\end{itemize}

\textbf{Example:}
\begin{itemize}
\item
The set of all integers is denoted as $\mathrm{S}=$ fa: a is an integer Here, $7 \in \mathrm{S}$ but $1 / 7 \notin \mathrm{S}$
\item
The set of all odd numbers denoted as $\mathrm{S}=\{\mathrm{a} ;$ a is not divisible by 23 Here, $7 \in \mathrm{S}$ but $8 \notin \mathrm{S}$
\item
The set of prime numbers less than 100 is denoted as $\mathrm{S}=\{$ a: a is prime and less than 100 ? Here, $23 \in \mathrm{S}$ but 98 or $101 \notin \mathrm{S}$
\end{itemize}

\textbf{1.2.1 Subset}
Let there be two sets $\mathrm{S}$ and $\mathrm{S}_{1} \mathrm{~S}_{1}$ is said to be a subset of $\mathrm{S}$

if every element $\mathrm{S},$ is an element of $\mathrm{S}$.Symbolically,it is denoted as $\mathrm{S}_{i} \subset \mathrm{S}$

The reverse of a subset is the superset.In the previous example,$\mathrm{S}$ is the

superset of $\mathrm{S}_{1}$


\textbf{Example:}
\begin{itemize}
\item
Let $Z$ be the set of all integers.$E$ is the set of all even numbers. All even numbers are natural numbers.So,it can be denoted as $E \subset Z$.
\item
Let $S$ be the set of the numbers divisible by $6,$ where $T$ is the set of numbers divisible by 2.Property says that if a number is divisible by $6,$ it must be divisible by 2 and $3.$ So,it can be denoted as $\mathrm{S} \subset \mathrm{T}$.
\end{itemize}

\textbf{1.2.2 Finite and Infinite Set}
A set is said to be finite if it contains no

element or a finite number of elements.Otherwise,it is an infinite set.


\textbf{Example:}
\begin{itemize}
\item
Let $\mathrm{S}$ be the set of one digit integers greater than $1,\mathrm{~S}$ is finite as its number of elements is 8.
\item
Let P be the set of all prime numbers.T is infinite,as the number of prime numbers is infinite.
\end{itemize}
\end{document}
