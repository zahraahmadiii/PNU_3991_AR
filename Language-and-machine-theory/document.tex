\documentclass [12pt]{beamer}
\usepackage{xcolor}	
\usepackage{tikz}
\usetheme{Warsaw}
\useoutertheme{infolines}
\usepackage{ragged2e}
\usepackage{amsmath}
\usepackage{amssymb}
\begin{document}

\section*{mesal 1.1 , 1.2 }
\subsection*{zahra ahmadi  }	
\begin{frame}
\justifying
Example 1.1 ) A relation $R$ is defined on set (I) of all integers, by aRb, if and only if ab $>0,$ for all $\mathrm{a}, \mathrm{b} \in \mathrm{I} .$ Examine if $\mathrm{R}$ is (a) reflexive, (b) symmetric, and (c) transitive.
\begin{flushleft}
Solution:
\end{flushleft}
a)  Let $a, b \in I$. then  $aa>0,$ holds good  if $b a>0$ if a = 0,then a.a=0.
therefor, aRa does not hold for all a in I. So, R is not reflexive.
\begin{flushleft}
b) Let $a, b \in I$. If $a b>0,$ then $b a>0$ also. That is, $a R b \Rightarrow b R a .$ So, $R$ is symmetric.
\end{flushleft}
c ) Let $a, b \in I$. i let aRb and bRc both hold good.It can be said 
\begin{flushleft}
ab $>0 and  bc $>0  . if we  multiply these two,then(ab)(bc)$>0,\mathrm{ab}^{2} \mathrm{c}>0 .$ We know $\mathrm{b}^{2}$ is always $>0 .
It can be said $ clearly that $\mathrm{ac}>0$.
Thus, aRb and bRc $\Rightarrow$ aRc. So, R is transitive.
\end{flushleft}
\end{frame}
\begin{frame}	
\justifying
Example 1.2 ) A relation $\mathrm{R}$ is defined on a set of integers (I) by aRb if a $-\mathrm{b}$ is divisible by $3,$ for  $\mathrm{a}, \mathrm{b} \in \mathrm{I} .$ Examine if $\mathrm{R}$ is (a) reflexive (b) symmetric, and (c) transitive.
\begin{flushleft}
Solution:
\end{flushleft}

a) Let a $\in 1 .$ Then, $\mathrm{a}-\mathrm{a}$ is divisible $3 .$ Therefore, aRa holds good for all a $\in 1 .$ So the relation $\mathrm{R}$ is reflexive.
\begin{flushleft}
b) Let $a, b \in I$ and aRb hold good. It means $a-b$ is divisible by $3 .$ If it is true, then $b-a$ is also divisible by 3 .
\end{flushleft} 
\begin{flushleft}
(Let $a=6, b=3, a-b=3,$ divisible by $3, b-a=-3,$ which is also divisible by 3 ). So, bRa holds good. Therefore, $\mathrm{R}$ is symmetric.
\end{flushleft}
\begin{flushleft}
c) Let $a, b, c \in I$ and $a R b, b R c$ hold good. It means $a-b$ and $b-c$ both are divisible by $3 .$ Therefore, $(a-c)=(a-b)+(b-c)$ is divisible by 3.
(Let $a=18, b=12, c=6, a-b=6, b-c=6 .$ Both are divisible by $3 . a-c=12$ is also divisible by $3 .$ So, aRc holds good. Therefore, $\mathrm{R}$ is transitive.
\end{flushleft}
\end{frame}
\end{document}

