\documentclass[10pt]{article}
\usepackage{multicol}
\usepackage{xcolor}
\usepackage{graphicx}
\linespread{1.35}
\usepackage{amsmath}
\usepackage{color}
\usepackage{tikz}
\usetikzlibrary{arrows,automata}

\begin{document}

\begin{flushright}
 \texttt{Basic Terminology} \hspace*{0.1cm}\textbf{$|$} \hspace*{0.1cm} \textbf{3}\hspace*{0.1cm}
\end{flushright}

\vspace*{1cm}
\large{\textbf{1.2.3 Equal}}

\vspace*{0.1cm}
Two sets S and $S_1$ are said to be equal if S is a subset of $S_1$ and S1 is a subset of $S$.

\large{\textbf{1.2.4 Algebraic Operations on Sets}}

\begin{itemize}
  \item \textbf{Union:} If there are two sets $A$ and $B$, then their union is denoted by $A \cup B$.\\
  Let $A$ =$ \{2, 3, 4\}$ and $B$ =$ \{3, 5, 6\}$. Then, $A \cup B$ =$ \{2, 3, 4, 5, 6\}$. In general, $A \cup B$ =$ \{x | x \in A$ or $x \in B\}$ .\\
 \hspace*{0.2cm} Diagrammatically, the union operation on two sets can be represented as shown in Fig. 1.1. This diagrammatic representation of sets is called the Venn diagram.
\end{itemize}


\begin{center}
\section{picture}
\includegraphics[width=3cm,height=3cm]{3-1.png}
\end{center}

\begin{itemize}
  \item \textbf{Intersection:} If there are two sets A and B, then their intersection is
denoted by $A \cap B$. Let $A$ $= \{2, 3, 4\}$ and $B$ $= \{3, 4, 5, 6\}$. Then, $A \cap B$ $=
\{3, 4\}$. In general, $A \cap B$ $= \{x | x \in A$ and $x \in B\}$.
The Venn representation of the intersection operation on two sets can be represented as shown in Fig. 1.2.
\end{itemize}
\begin{center}
\section{picture}
\includegraphics[width=3cm,height=3cm]{3-2.png}
\end{center}

\begin{itemize}
  \item \textbf{Difference:} If there are two sets A and B, then their difference is denoted by $A - B$. $Let A$ $= \{2, 3, 4, 5\}$ and $B$ $= \{3, 4\}$. Then, $A - B$ $= \{2, 5\}$. In general, $A - B$ $= \{x | x \in A and x \in B\}$.\\
   \hspace*{0.2cm} The Venn representation of the difference operation on two sets can be represented as shown in Fig. 1.3.

\end{itemize}

\begin{center}
\section{picture}
\includegraphics[width=3cm,height=3cm]{3-3.png}
\end{center}

\begin{itemize}
  \item \textbf{Complementation:} The complement of a set A, which is a subset of a large set U is denoted by AC or $A'$, defined by $A'$ $= \{x \in U: x A\}$.\\
\hspace*{0.2cm} The Venn representation of the complement operation is given in Fig. 1.4.

\begin{center}
\section{picture}
\includegraphics[width=3cm,height=3cm]{3-4.png}
\end{center}
\end{itemize}

\begin{itemize}
  \item \textbf{Cartesian product:} If there are two sets A and B, then their Cartesian
product is denoted by $A \times B$. Let $A$ $= \{2, 3, 4, 5\}$ and $B$ $= \{3, 4\}$. Then,
$A \times B$ $= \{(2, 3), (2, 4), (3, 3), (3, 4),(4, 3), (4, 4), (5, 3), (5, 4)\}$. In general,
$A \times B$ $= \{(a, b)| a \in A and b \in B\}$.
  \item \textbf{Power set:} The power set of a set A is the set of all possible subsets of A.
Let $A$ $= \{a, b\}$. Then, the power set of A is $\{({\O}), (a), (b), (a, b)\}$. For a set
of elements n, the number of elements of the power set of A is $2^n$.
\end{itemize}

\large{\textbf{1.2.5 Properties Related to Basic Operation}}

\vspace*{0.2cm}
Some properties related to basic operations on set are as follows:

\vspace*{0.2cm}
\begin{itemize}
  \item $A \cup {\O}$ $= A, A \cap {\O} = A$ $({\O}$ is called null set)
  \item $A \cup U$ $= U$, $A \cap U =$ A (where $A \subset U)$
  \item $A \cup A$ $= A, A \cap A = A$ (idempotent law)
\end{itemize}

\end{document}







